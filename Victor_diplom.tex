\documentclass[10pt,a4paper]{article}
\usepackage[utf8]{inputenc}
\usepackage[russian]{babel}
\usepackage{amsmath}
\usepackage{amsfonts}
\usepackage{amssymb}
\begin{document}
\begin{center}
	Московский физико-технический институт (Государственный университет)\\
Факультет управления и прикладной математики\\
	\vspace{1cm} Моделирование многофазных реагирующих фильтрационных течений с 	равновесными химическими реакциями.\\
	\vspace{1cm}Выпускная квалификационная работа на степень бакалавра\\
			\vspace{1cm}студента 373 группы\\
	\vspace{1cm}Гринина Виктора Олеговича
	\vspace{4cm}
\end{center} 
\par\textbf{Введение}\\
В данной работе рассматриваются многофазные фильтрационные течения, в которых наряду с реакциями с конечной кинетикой присутствуют равновесные реации. Для описания процесса используются система уравнений, описывающая процесс установления химического равновесия, и система уравнений, описывающая  многофазное фильтрационное течение. Системы решаются последовательно. Для решения 2-й используется симулятор многофазных фильтрационных течений. Цель работы заключалась в написании модуля для решения 1-й системы и добавлении его к имеющемуся программному комплексу, для проведения численных экспериментов.
\par\textbf{Химические уравнения}\\
В случае равновесных химических реакций вместо скорости реакции имеется равновесное соотношение $$F(N_i) = 0$$\\
Например, для случаю диссоциации 
	$$\sum{\nu_i X_i} \rightleftharpoons 0$$\\
в качестве такой функции можно взять $$F = \ln{K} + \sum{\nu_i \ln{N_i}}$$\\
Пусть в силу  некоторых причин данное равновесие оказалось нарушено. Пусть конценртации $N_i^0$. Тогда из-за данной реакции концентрации изменяются по закону $$\Delta N_i = \xi \nu_i,$$ где $\xi$ - величина, характеризующая глубину реакции, одинаковая для всех участвующих компонент. \\
Задача восстановления равновесия заключается в поиске такого значени $\xi$, что $F(N_i) = 0$ . При этом, можно сделать очевидное обощение на случай нескольких реакций\\
 $$N_i = N_i^0 + \sum{\xi_j\nu_{i,j}}$$ $$F_j(N_i)=\ln(K_j) + \sum{\nu_{i,j}ln{N_i}} = 0$$ 
\par\textbf{Конкретные химические реакции}\\ 
$$R1: OH^{-} + H^{+} \rightleftharpoons H_2O,$$
$$R2: HCO_3^{-}+H^{+} \rightleftharpoons H_2O + CO_2,$$
$$R3: CaCO_3 + 2H^{+} \rightleftharpoons H_2O + CO_2 + Ca^{2+}.$$\\
Эта  система может быть записана в матрично-векторной форме $SY \rightleftharpoons 0$, где
$$S =  \begin{vmatrix}
-1 &0 &0  &1 &0 &-1 &0\\
0 &-1 &0  &1 &1 &-1 &0\\
0  &0 &0  &1 &1 &-2 &1
		\end{vmatrix},
  Y = \begin{vmatrix}
  OH^{-}\\
  HCO_3^{-}\\
  CaCO_3\\
  H_2O\\
  CO_2\\
  H^{+}\\
  Ca^{2+}
  \end{vmatrix}$$\\
Или в виде таблицы Мореля\\
$$\begin{array}{|c|ccccc|}
\hline\\
		&H_20	&H^{+}	&CO_2	&Ca^{2+}	&\log{K}\\
\hline\\
OH^{-}		&1		&-1		&0		  &0		&-14\\
\hline\\
HCO_3^{-}	&1		&-1		&1		  &0		&-5.928\\
\hline\\
CaCO_3		&1		&-2		&1		  &1		&-8.094\\
\hline
\end{array}$$
\par\textbf{Численный метод и программный модуль}\\
\hspace*{0.5cm}Для решения системы, которая описывает установление химического равновесия, использовался метод Ньютона. Сходимость метода зависит от вида, в котором записана система. \\
\hspace*{0.5cm} Запишем приведённую раньше систему в матричной форме$$\bar{F} = \ln{\bar{K}} + V^T \ln{(\bar{N}^0 + V\xi)}$$
\hspace*{0.5cm}Продифференцируем эту функцию по $\xi$
$$\frac{\partial \bar{F}}{\partial{\bar{\xi}}} = V^Tdia g^{-1}(\bar{N}^0 + V\bar{\xi})V$$
\hspace*{0.5cm}Метод Ньютона принимает вид
$$\bar{\xi}^{k+1} = \bar{\xi}^{k} - \alpha^{(k)}[V^Tdia g^{-1}(\bar{N}^0 + V\bar{\xi})V]^{-1}(\ln{\bar{K}} + V^T \ln{(\bar{N}^0 + V\bar{\xi})})$$
При расчёте данного метода на компьютере возникает несколько проблем. Нужно выбирать начальное приближение $\bar{\xi}^0$ так, чтобы выражение под логарифмом было положительным $\bar{N}^0 + V\bar{\xi}\geq 0$. Для чего нужно решать ситему неравенств. Кроме того на каждой итерации следует выбирать параметр $\alpha^{(k)} \in [0,1]$ так, чтобы это неравенство не нарушалось. Однако даже тогда метод может не работать. Концентрации реагирующих веществ могут отличаться на порядки. В результате, некоторые вещества практически исчезают в ходе выполнения алгоритма. Так как все вычисления производятся на компьютере с конечной точностью, то в какой-то момент дальнейший счёт становится невозможным.\\
\hspace*{0.5cm}Перепишем систему в другом виде, для этого введём дополнительные переменные $$\bar{p} = \ln{(\bar{N}^0 + V\bar{\xi})}$$ 
\hspace*{0.5cm}При этом получаем рассширенную систему
$$\begin{cases} 
	\bar{F} = ln{\bar{K}} + V^T\bar{p}=0,\\
	\exp(\bar{p})=\bar{N}^0 + V\bar{\xi};
	
\end{cases}$$
$$\begin{cases} 
	ln{\bar{K}} + V^T\bar{p}=0,\\
	\bar{N}^0 + V\bar{\xi} - \exp(\bar{p}) = 0;
\end{cases}$$
Эта система полностью эквивалентна исходной. Её можно записать в виде
\hspace*{3.5 cm} $\bar{\Phi}(\bar{x}) = 0$, где $ \bar{x} = [\bar{\xi}, \bar{p}]$\\
Тогда метод Ньютона принимает вид\\
$$\bar{x}^{k+1} = \bar{x}^{k} - \alpha^{(k)}(\frac{\partial\bar{\Phi}}{\partial \bar{x}})^{-1}\bar{\Phi}$$
 где $$(\frac{\partial\bar{\Phi}}{\partial \bar{x}})^{-1} = \begin{vmatrix}
 0 & V^T \\
 V & diag({-e^{\bar{p}}})
\end{vmatrix}  $$
Использование метода Ньтона переписанного в такой форме, уже не встречает проблем, характерных предыдущей версии. Как показывает эксперимент он сходится из любого начального приближения, при любых начальных концентрациях. Кроме того в данном случае можно выбрать $\alpha^{(k)} = 1$, что обеспечивает большую скорость сходимости.
\par\textbf{Результаты}\\ 

\end{document}