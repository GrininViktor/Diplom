\documentclass[12pt,a4paper]{article}
\usepackage[utf8]{inputenc}
\usepackage[russian]{babel}
\usepackage{amsmath}
\usepackage{amsfonts}
\usepackage{amssymb}
\usepackage[left=2.5cm, right=2cm, bottom=1.5cm, top=2cm]{geometry}
\usepackage{indentfirst}

\newcommand{\pd}[2]{\frac{\partial #1}{\partial #2}}
\newcommand{\dpd}[2]{\dfrac{\partial #1}{\partial #2}}

\newcommand{\pdd}[2]{\frac{\partial^2 #1}{\partial #2^2}}
\newcommand{\pddd}[3]{\frac{\partial^2 #1}{\partial #2\partial #3}}

\renewcommand{\arraystretch}{1.2}

\let\dividesymbol\div
\renewcommand{\div}{\operatorname{div}}
\newcommand{\grad}{\operatorname{grad}}
\newcommand{\rot}{\operatorname{rot}}
\newcommand{\const}{\operatorname{const}}
\newcommand{\diag}{\operatorname{diag}}
\renewcommand{\vec}[1]{\boldsymbol{\mathbf{#1}}}
\newcommand{\ten}[1]{\mathbf{#1}}
\newcommand{\cutefrac}[2]{{}^{#1}\mkern-5mu/{\!}_#2}
\newcommand{\half}{{\cutefrac{1}{2}}}
\renewcommand{\leq}{\leqslant}
\renewcommand{\geq}{\geqslant}

\begin{document}
\begin{center}
	Московский физико-технический институт (Государственный университет)\\
Факультет управления и прикладной математики\\
	\vspace{1cm} Моделирование многофазных реагирующих фильтрационных течений с 	равновесными химическими реакциями.\\
	\vspace{1cm}Выпускная квалификационная работа на степень бакалавра\\
			\vspace{1cm}студента 373 группы\\
	\vspace{1cm}Гринина Виктора Олеговича
	\vspace{4cm}
\end{center} 

\clearpage
\tableofcontents

\clearpage
\section{Введение}

В данной работе рассматриваются многофазные фильтрационные течения, в которых наряду с реакциями с конечной кинетикой присутствуют равновесные реакции. Для описания процесса используются система уравнений, описывающая процесс установления химического равновесия, и система уравнений, описывающая  многофазное фильтрационное течение. Системы решаются последовательно. Для решения 2-й используется симулятор многофазных фильтрационных течений. Цель работы заключалась в написании модуля для решения 1-й системы и добавлении его к имеющемуся программному комплексу, для проведения численных экспериментов.

\section{Математическая модель}

\subsection{Химические уравнения}

В случае равновесных химических реакций вместо скорости реакции имеется равновесное соотношение
$$F(N_i) = 0$$
Например, для случаю диссоциации 
$$\sum{\nu_i X_i} \rightleftharpoons 0$$
в качестве такой функции можно взять $$F = \ln{K} + \sum{\nu_i \ln{N_i}}$$
Пусть в силу  некоторых причин данное равновесие оказалось нарушено. Пусть концентрации $N_i^0$. Тогда из-за данной реакции концентрации изменяются по закону $$\Delta N_i = \xi \nu_i,$$ где $\xi$ - величина, характеризующая глубину реакции, одинаковая для всех участвующих компонент.

Задача восстановления равновесия заключается в поиске такого значения $\xi$, что $F(N_i) = 0$. При этом, можно сделать очевидное обобщение на случай нескольких реакций
\begin{gather*}
N_i = N_i^0 + \sum_{j=1}^{...} \xi_j\nu_{i,j}\\
F_j(N_i)=\ln(K_j) + \sum_{i=1}^{...} \nu_{i,j}\ln{N_i} = 0
\end{gather*}
 
\subsection{Конкретные химические реакции}

\newcommand{\OHm}{\text{OH}^-}
\newcommand{\Hp}{\text{H}^+}
\newcommand{\WAT}{\text{H}_2\text{O}}
\newcommand{\CARB}{\text{CO}_2}
\newcommand{\Catwop}{\text{Ca}^{2+}}
\newcommand{\Calcite}{\text{CaCO}_3}
\newcommand{\HCO}{\text{HCO}_3}

\begin{align*}
R1:&\quad \OHm + \Hp \rightleftharpoons \WAT,\\
R2:&\quad \HCO +\Hp \rightleftharpoons \WAT + \CARB,\\
R3:&\quad \Calcite + 2\Hp \rightleftharpoons \WAT + \CARB + \Catwop.
\end{align*}

Эта  система может быть записана в матрично-векторной форме $SY \rightleftharpoons 0$, где
$$S =  \begin{Vmatrix}
-1 &0 &0  &1 &0 &-1 &0\\
0 &-1 &0  &1 &1 &-1 &0\\
0  &0 &0  &1 &1 &-2 &1
		\end{Vmatrix},
  Y = \begin{Vmatrix}
  \OHm\\
  \HCO\\
  \Calcite\\
  \WAT\\
  \CARB\\
  \Hp\\
  \Catwop
  \end{Vmatrix}$$\\
Или в виде таблицы Мореля\\
$$\begin{array}{|c|cccc|c|}
\hline
		&\WAT	&\Hp	&\CARB	&\Catwop	&\lg {K}\\
\hline
\OHm		&1		&-1		&0		  &0		&-14\\
%\hline
\HCO	&1		&-1		&1		  &0		&-5.928\\
%\hline
\Calcite		&1		&-2		&1		  &1		&-8.094\\
\hline
\end{array}$$

\section{Численный метод и программный модуль}

Для решения системы, которая описывает установление химического равновесия, использовался метод Ньютона. Сходимость метода зависит от вида, в котором записана система.

Запишем приведённую раньше систему в матричной форме$$\vec{F} = \ln{\vec{K}} + V^T \ln{(\vec{N}^0 + V\vec \xi)}$$
Продифференцируем эту функцию по $\vec \xi$
$$\frac{\partial \vec{F}}{\partial{\vec{\xi}}} = V^T\diag^{-1}(\vec{N}^0 + V\vec{\xi})V$$

Метод Ньютона принимает вид
$$\vec{\xi}^{k+1} = \vec{\xi}^{k} - \alpha^{(k)}[V^T\diag^{-1}(\vec{N}^0 + V\vec{\xi})V]^{-1}(\ln{\vec{K}} + V^T \ln{(\vec{N}^0 + V\vec{\xi})})$$
При расчёте данного метода на компьютере возникает несколько проблем. Нужно выбирать начальное приближение $\vec{\xi}^0$ так, чтобы выражение под логарифмом было положительным $\vec{N}^0 + V\vec{\xi}\geq 0$. Для чего нужно решать систему неравенств. Кроме того на каждой итерации следует выбирать параметр $\alpha^{(k)} \in [0,1]$ так, чтобы это неравенство не нарушалось. Однако даже тогда метод может не работать. Концентрации реагирующих веществ могут отличаться на порядки. В результате, некоторые вещества практически исчезают в ходе выполнения алгоритма. Так как все вычисления производятся на компьютере с конечной точностью, то в какой-то момент дальнейший счёт становится невозможным.

Перепишем систему в другом виде, для этого введём дополнительные переменные $$\vec{p} = \ln{(\vec{N}^0 + V\vec{\xi})}$$ 

При этом получаем рассширенную систему
$$\begin{cases} 
	\vec{F} = \ln{\vec{K}} + V^T\vec{p}=0,\\
	\exp(\vec{p})=\vec{N}^0 + V\vec{\xi};
	
\end{cases}$$
$$\begin{cases} 
	\ln{\vec{K}} + V^T\vec{p}=0,\\
	\vec{N}^0 + V\vec{\xi} - \exp(\vec{p}) = 0;
\end{cases}$$
Эта система полностью эквивалентна исходной. Её можно записать в виде
$\vec{\Phi}(\vec{x}) = 0$, где $ \vec{x} = [\vec{\xi}, \vec{p}]$

Тогда метод Ньютона принимает вид\\
$$\vec{x}^{k+1} = \vec{x}^{k} - \alpha^{(k)}\left(\pd{\vec{\Phi}}{\vec{x}}\right)^{-1}\vec{\Phi}(\vec x^k)$$
 где $$\pd{\vec{\Phi}}{\vec{x}} = \begin{Vmatrix}
 0 & V^T \\
 V & -\diag({\exp(\vec{p})})
\end{Vmatrix}  $$
Использование метода Ньютона переписанного в такой форме, уже не встречает проблем, характерных предыдущей версии. Как показывает эксперимент он сходится из любого начального приближения, при любых начальных концентрациях. Кроме того в данном случае можно выбрать $\alpha^{(k)} = 1$, что обеспечивает большую скорость сходимости.

\section{Результаты}

\end{document}