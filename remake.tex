\documentclass[12pt,a4paper]{article}
\usepackage[utf8]{inputenc}
\usepackage[russian]{babel}
\usepackage{amsmath}
\usepackage{amsfonts}
\usepackage{amssymb}
\usepackage[left=2.5cm, right=2cm, bottom=1.5cm, top=2cm]{geometry}
\usepackage{indentfirst}
\usepackage{tikz}

\newcommand{\pd}[2]{\frac{\partial #1}{\partial #2}}
\newcommand{\dpd}[2]{\dfrac{\partial #1}{\partial #2}}

\newcommand{\pdd}[2]{\frac{\partial^2 #1}{\partial #2^2}}
\newcommand{\pddd}[3]{\frac{\partial^2 #1}{\partial #2\partial #3}}

\renewcommand{\arraystretch}{1.2}

\let\dividesymbol\div
\renewcommand{\div}{\operatorname{div}}
\newcommand{\grad}{\operatorname{grad}}
\newcommand{\rot}{\operatorname{rot}}
\newcommand{\const}{\operatorname{const}}
\newcommand{\diag}{\operatorname{diag}}
\renewcommand{\vec}[1]{\boldsymbol{\mathbf{#1}}}
\newcommand{\ten}[1]{\mathbf{#1}}
\newcommand{\cutefrac}[2]{{}^{#1}\mkern-5mu/{\!}_#2}
\newcommand{\half}{{\cutefrac{1}{2}}}
\renewcommand{\leq}{\leqslant}
\renewcommand{\geq}{\geqslant}

\begin{document}
\textbf{{\large Билет 11}}
\section{Задача}
Для уравнения $u_{tt}=c^2u_{xx} + f$ составить схему на шаблоне \quad\tikz {
\draw (0,0) --  ++(1,0) -- +(1,0) -- +(0,0)-- +(0,-1) -- +(-1,-1)-- +(1,-1)--+(0,-1)--+(0,1);
\fill[black] (0,0) circle (1mm);<br>
\fill[black] (1,0) circle (1mm);<br>
\fill[black] (2,0) circle (1mm);<br>
\fill[black] (0,-1) circle (1mm);<br>
\fill[black] (1,-1) circle (1mm);<br>
\fill[black] (2,-1) circle (1mm);<br>
\fill[black] (1,1) circle (1mm);<br>
} 

\begin{flushleft}
и исследовать аппроксимацию и устойчивость. Устойчивость исследовать 2-мя способами: методом гармоник и операторным методом, предварительно записав схему в канонической форме. 
\end{flushleft}

\subsection{Рассмотрим вопрос об аппроксимации.}

$$u_{tt} = c^2u_{xx} + f \Rightarrow \frac{u_i^{j+1}- 2u_i^j + u_i^{j-1}}{\tau^2} = c^2\left( \sigma\frac{u_{i+1}^j- 2u_i^j + u_{i-1}^j}{h^2} + (1-\sigma)\frac{u_{i+1}^{j-1}- 2u_i^{j-1} + u_{i-1}^{j-1}}{h^2}\right)  + \varphi_i^j$$
В более компактном виде 
$$ u_{\bar{t}t} = c^2(\sigma u_{\bar{x}x} + (1-\sigma)\check{u}_{\bar{x}x})  + \varphi \hspace{0.5cm}(*)$$
Разложим все вхождения функции в ряд Тейлора в окрестности точки (i, j).

\begin{flushleft}
Получим
\end{flushleft}
$$ u_{tt} + o(\tau^2) = c^2(\sigma u_{xx} + (1-\sigma)\check{u}_{xx} + o(h^2)) $$
или $$ u_{tt} + o(\tau^2) = c^2(\sigma u_{xx} + (1-\sigma)u_{xx} - (1-\sigma)u_{xxt} + o(h^2 + \tau^2)) $$

$$u_{tt} - c^2u_{xx} = (\sigma - 1)u_{xxt} + o(h^2 + \tau^2) = o(h^2 + \tau)\text{ в случае }\sigma\neq 1$$
(при $\sigma =1$ получаем схему крест, имеющую аппроксимацию $o(h^2 + \tau^2)$, но это всё таки другой шаблон)

\subsection{Рассмотрим вопрос об устойчивость схемы.}

\subsubsection{Метод разделения переменных.}
	 
Будем искать решение численной задачи в виде $$u = T(t)*X(x)$$
Подставляя в $(*)$ и полагая $\varphi = 0$, получаем $$X\frac{\hat{T}-2T +\check{T}}{\tau^2} = c^2\Lambda(\sigma TX+(1-\sigma)\check{T}X)$$ $$\frac{\hat{T}-2T +\check{T}}{c^2\tau^2(\sigma T - (1-\sigma)\check{T})} = \frac{\Lambda X}{X} = -\lambda $$
На X имеем задачу Штурма-Лиувиля(доп.$ X(0) = X(1) = 0$), решение которой хорошо известно. 

Для временной функции полагаем $$\hat{T} = qT , \quad\check{T} = \frac{1}{q}T$$
Покажем, что при определённом выборе $\sigma$ выполнено $|q_{k}|\leqslant 1$, т.е схема устойчива.


 На q имеем уравнение
 $$q - 2 + \frac{1}{q} = -\lambda c^2\tau^2(\sigma +(1-\sigma)\frac{1}{q}),$$
					 $$q^2-2q+1 = -\lambda c^2\tau^2(\sigma(q-1) + 1),$$
					 $$q^2+(\lambda c^2\tau^2\sigma-2)q+1+(1-\sigma)\lambda c^2\tau^2 = 0,$$
					 $$D = (\lambda c^2\tau^2\sigma-2)^2 -4 - 4(1-\sigma)\lambda c^2\tau^2 = $$
					 $$= (\lambda c^2\tau^2\sigma)^2 - 4\lambda c^2\tau^2; $$


Если $\sigma \leqslant\frac{2}{\sqrt{\lambda}\tau c} $, то $ D\leqslant 0$, и мы имеем два комплексно-сопряженных корня, по модулю равных $\sqrt{1+(1-\sigma)\lambda c^2\tau^2}$.

Отсюда получаем $ |q_{k}| \leqslant 1 $, при  $ 1\leqslant\sigma \leqslant \frac{2}{\sqrt{\lambda_{k}}\tau c}$ и, учитывая $\lambda_k \leqslant \frac{4}{h^2}$, условие устойчивости $1\leqslant\sigma \leqslant \frac{h}{\tau c}$.

Т.о мы показали, что при числе Куранта $k =\frac{c\tau}{h}\leqslant1$, можно так выбрать $\sigma$, что схема будет устойчивой (при $\sigma = 1$ опять же получаем устойчивую схему крест)

\subsubsection{Метод гармоник.}
Вводим в  исходном уравнении параметр $\delta = k^2$(квадрату числа Куранта), собираем коэффициенты при одних и тех же значениях функции и отбрасываем правую часть.
$${u_i^{j+1}- 2u_i^j + u_i^{j-1}} = \delta(\sigma(u_{i+1}^j- 2u_i^j + u_{i-1}^j) + (1-\sigma)(u_{i+1}^{j-1}- 2u_i^{j-1} + u_{i-1}^{j-1})) $$
 
$$u_i^{j+1} - \delta\sigma u_{i+1}^j + 2(\delta\sigma - 1)u_i^j - \delta\sigma u_{i-1}^j - \delta (1-\sigma)u_{i+1}^{j-1} + (1+2\delta(1-\sigma))u_i^{j-1} - \delta (1-\sigma)u_{i-1}^{j-1}= 0$$
подставляя гармоники $y_m^n = q^n e^{i\varphi m}$ получаем
$$q^2 - \delta\sigma q e^{i\varphi} + 2(\delta\sigma - 1)q - \delta\sigma q e^{-i\varphi} - \delta (1-\sigma)e^{i\varphi} + (1+2\delta(1-\sigma)) - \delta (1-\sigma)e^{-i\varphi}= 0$$
$$q^2 - q(\delta\sigma(e^{i\varphi} + e^{-i\varphi}) - 2(\delta\sigma-1)) + 1 - \delta(1-\sigma)(e^{i\varphi} + e^{i\varphi} - 2) = 0$$
$$q^2 - 2q(\delta\sigma(\cos{\varphi} - 1) + 1) + 1 - 2\delta(1-\sigma)(\cos{\varphi} - 1) = 0$$
$$\frac{D}{4} = 1 + 2\delta\sigma(\cos{\varphi} - 1) + \delta^2\sigma^2(\cos{\varphi} - 1)^2 - 1 - 2\delta\sigma(\cos{\varphi} - 1) + 2\delta(\cos{\varphi} - 1) = \delta^2\sigma^2(\cos{\varphi} - 1)^2 + 2\delta(\cos{\varphi} - 1)$$
Учитывая, что $\cos{\varphi} - 1 \leq 1$ имеем
$$D \leq 0 \text{ при всех }\varphi \text{, если } \sigma^2 \leq \frac{1}{\delta}$$
$$\text{Тогда корни комплексно сопряжённые, а модуль их меньше 1, если } \sigma \geq 1$$
В итоге, как и в методе разделения переменных получаем, что при $1\leq\sigma \leq \frac{1}{k}$ схема устойчива. А существуют такие $\sigma$ при $ k \leq 1$

\subsubsection{Операторный метод.}
$$u_i^{j+1} - \delta\sigma u_{i+1}^j + 2(\delta\sigma - 1)u_i^j - \delta\sigma u_{i-1}^j - \delta (1-\sigma)u_{i+1}^{j-1} + (1+2\delta(1-\sigma))u_i^{j-1} - \delta (1-\sigma)u_{i-1}^{j-1}= \tau^2\varphi_i^j$$

Или в матричном виде 

$$B_0\hat{u}+B_1u+B_2\check{u} = \tau\varphi,$$

где $$ B_0 = \tau^{-1}E,\quad B_1 = \tau^{-1}\begin{vmatrix}
2(\delta\sigma-1) & -\delta\sigma & 0 & 0&.\\
-\delta\sigma & 2(\delta\sigma-1) & -\delta\sigma&0&.\\
.&.&.&.&.\\
.&.&0&-\delta\sigma & 2(\delta\sigma - 1)
\end{vmatrix}
$$
$$B_2 = \tau^{-1}\begin{vmatrix}
1 + 2\delta(1-\sigma) & -\delta(1-\sigma) & 0 & 0&.\\
-\delta(1-\sigma)  & 1 + 2\delta(1-\sigma)& -\delta(1-\sigma) &.\\
.&.&.&.&.\\
.&.&0&-\delta(1-\sigma) & 1 + 2\delta(1-\sigma)
\end{vmatrix}
$$

Перепишем трёхслойную схему в канонической форме
$$B\frac{\hat{u} - \check{u}}{2\tau} + R(\hat{u} - 2u + \check{u}) + Au = \varphi$$\\

Здесь $$B = B_0 - B_2 = \frac{1}{\tau}\begin{vmatrix}
- 2\delta(1-\sigma) & \delta(1-\sigma) & 0 & 0&.\\
\delta(1-\sigma)  & - 2\delta(1-\sigma)& \delta(1-\sigma) &.\\
.&.&.&.&.\\
.&.&0&\delta(1-\sigma) & - 2\delta(1-\sigma)
\end{vmatrix}$$
$$R = \frac{1}{2\tau}(B_0 + B_2) = \frac{1}{2\tau^2} \begin{vmatrix}
2 + 2\delta(1-\sigma) & -\delta(1-\sigma) & 0 & 0&.\\
-\delta(1-\sigma)  & 2 + 2\delta(1-\sigma)& -\delta(1-\sigma) &.\\
.&.&.&.&.\\
.&.&0&-\delta(1-\sigma) & 2 + 2\delta(1-\sigma)
\end{vmatrix}
$$
$$A = \frac{1}{\tau}(B_0 + B_1 + B_2) = \frac{1}{\tau^2} \begin{vmatrix}
2\delta & -\delta& 0 & 0&.\\
-\delta & 2\delta & -\delta&0&.\\
.&.&.&.&.\\
.&.&0&-\delta & 2\delta
\end{vmatrix}
$$



 $$R - \frac{1}{4}A = \frac{1}{2\tau^2}\begin{vmatrix}
2 + \delta - 2\delta\sigma & -\frac{\delta}{2} + \delta\sigma & 0 & 0&.\\
-\frac{\delta}{2} + \delta \sigma & 2 + \delta - 2\delta\sigma  & -\frac{\delta}{2} + \delta\sigma & .\\
.&.&.&.&.\\
.&.&0&-\frac{\delta}{2} + \delta\sigma & 2 + \delta - 2\delta\sigma 
\end{vmatrix}$$


Хотим, чтобы выполнялись условия теоремы об устойчивости трёхслойной схемы по начальным данным.

Матрицы $A$ и $R$ являются самосопряжёнными $A=A^{*}, R=R^{*}$.


\begin{flushleft}
 Потребуем, чтобы
 
1. $A > 0$

2. $B\geq 0$

3. $R > 0$

4. $R - \frac{1}{4}A > 0$ 

\end{flushleft}
Для выполнения этих требований необходимо и достаточно(доказательство можно найти в Приложении к работе), чтобы были справедливы следующие неравенства(условия диагонального преобладания для матриц + положительность диагональных элементов).
\begin{equation*}
 \begin{cases}
   2\geq2, & (A)
   \\
   -1+\sigma\geq0,  &(B)
   \\
   1+\delta(1-\sigma) \geq \delta(\sigma - 1), & (R)
   \\
   2 + \delta - 2\delta\sigma \geq -\delta + 2\delta\sigma. & (R-\frac{1}{4}A)
 \end{cases}
\end{equation*}
Получаем достаточное условие  устойчивости схемы по начальным данным    $$ 1\leq\sigma\leq\frac{1}{2}+\frac{1}{2\delta}$$
которое можно удовлетворить при том же $k\leq1$, что и в методе разделения переменных. 

Кроме того при выполнение условий, записанных выше, схема оказывается устойчивой и по правой части. 

$$\|Y(t+\tau)\|\leq \|Y(\tau)\| + M_2\max_{\tau < t^{`} \leq t}{(\| \varphi(t^{`})\|_{A^{-1}} + \| \varphi_{\bar{t}}(t^{`})\|_{A^{-1}})}$$

\section{Задача.} 
Привести к каноническому виду схему 
$$ 2\gamma y_i^{j+1} = (\gamma-\frac{1}{2})(y_{i-1}^{j+1} + y_{i+1}^{j+1}) +\frac{1}{2}(y_{i-1}^j+ y_{i+1}^j), \quad \gamma = \frac{\tau}{h^2}\quad\quad\tikz {
\draw (0,0) --  ++(0,1) -- ++(2,0) -- +(0,-1);
\fill[black] (0,0) circle (1mm);<br>
\fill[black] (0,1) circle (1mm);<br>
\fill[black] (1,1) circle (1mm);<br>
\fill[black] (2,1) circle (1mm);<br>
\fill[black] (2,0) circle (1mm);<br>
}$$
и исследовать её устойчивость. Устойчивость исследовать 2-мя способами: методом гармоник и операторным методом, предварительно записав схему в канонической форме. Сформулировать корректные начальные и граничные условия для разностной задачи. Предложить метод решения разностной задачи и сформулировать достаточные условия его устойчивости.

\subsection{Рассмотрим устойчивость схемы.}

\subsubsection{Метод гармоник.}

Запишем уравнение для погрешностей решения
$$ 2\gamma \delta y_i^{j+1} = (\gamma-\frac{1}{2})(\delta y_{i-1}^{j+1} + \delta y_{i+1}^{j+1}) +\frac{1}{2}(\delta y_{i-1}^j + \delta y_{i+1}^j), \quad \gamma = \frac{\tau}{h^2}$$
Гармоники $\delta y_{n,q}^j$ и $\delta y_{n,q}^{j+1}$ на слоях $j$ и $j+1$ связаны соотношениями
$$\delta y_{n,q}^{j+1} = \lambda_q\delta y_{n,q}^j, \quad \delta y_{n,q}^j = C_q(t_j)e^{iqhn}$$
Подставляя в уравнение получаем
$$2\gamma\lambda_q = (\gamma - \frac{1}{2})(\lambda_q e^{-iqh} + \lambda_q e^{igh}) + \frac{1}{2}(e^{-iqh} + e^{iqh})$$
$$\lambda_q = \frac{e^{-iqh} + e^{iqh}}{4\gamma +(1 - 2\gamma)(e^{-iqh} + e^{iqh})} = \frac{\cos{qh}}{2\gamma +(1 - 2\gamma)\cos{qh}}$$
Потребуем, чтобы особенность функции $\frac{-2\gamma}{1-2\gamma}$ не лежала на $[-1;1]$. Для этого  $\gamma > \frac{1}{4}$.

\begin{flushleft}
 Т.к $\lambda_q(\cos{qh}) = \lambda_q(x)\text{ возрастает при } x\in [-1,1]$, 
 $$\lambda_q ^{'}(x) = \frac{2\gamma}{(2\gamma +(1-2\gamma)x)^2} > 0$$
\begin{flushleft}
получаем
\end{flushleft}
 \end{flushleft} $$\lambda_q \in [\frac{1}{1-4\gamma}, 1]$$
Условие устойчивости схемы выполнено, если 
$$\frac{1}{1-4\gamma}\geq -1, \quad \gamma > \frac{1}{4}$$
или $$\gamma = \frac{\tau}{h^2} \geq \frac{1}{2}$$

\subsubsection{Операторный метод}
Запишем схему в матричном виде
$$B_0\hat{y}+B_1 y = 0$$
где $$B_0 = \begin{vmatrix}
2\gamma & \frac{1}{2} - \gamma & 0 & 0&.\\
\frac{1}{2} - \gamma & 2\gamma  & \frac{1}{2} - \gamma & .\\
.&.&.&.&.\\
.&.&0&\frac{1}{2} - \gamma &2\gamma  
\end{vmatrix} \quad B_1 = \begin{vmatrix}
0 & -\frac{1}{2} & 0 & 0&.\\
-\frac{1}{2} & 0 & -\frac{1}{2} & .\\
.&.&.&.&.\\
.&.&0&-\frac{1}{2} & 0
\end{vmatrix}$$
или в канонической форме
$$B\frac{\hat{y}-y}{\tau} + Ay = 0$$
$$B = B_0, \quad A = \frac{B_0+B_1}{\tau} = \frac{1}{\tau}\begin{vmatrix}
2\gamma & - \gamma & 0 & 0&.\\
- \gamma& 2\gamma  & -\gamma & .\\
.&.&.&.&.\\
.&.&0& -\gamma &2\gamma  
\end{vmatrix}$$
$$B-\frac{\tau}{2}A = \frac{1}{2}\begin{vmatrix}
2\gamma & 1 - \gamma & 0 & 0&.\\
1 - \gamma & 2\gamma  & 1 - \gamma & .\\
.&.&.&.&.\\
.&.&0& 1 - \gamma &2\gamma  
\end{vmatrix}$$
Матрицы $A$ и $B$ являются самосопряжёнными $A=A^{*}, B=B^{*}$.
\begin{flushleft}

  Требуем выполнения

  1. $A>0$

  2. $B>0$

  3. $ B-\frac{\tau}{2}A > 0$

\end{flushleft}
Для этого должна быть справедлива систему неравенств
\begin{equation*}
 \begin{cases}
   2\geq2, & (A)
   \\
   \gamma \geq |\frac{1}{2} - \gamma|,  &(B)
   \\
   \gamma \geq |1 - \gamma|. & (B-\frac{\tau}{2}A)
   
 \end{cases}
\end{equation*}
откуда получаем
$$\gamma\geq \frac{1}{2}$$
Т.о при $\gamma\geq \frac{1}{2}$ схема устойчива по начальным данным в норме операторов $A$ и $B$. 
Это условие совпадает с полученными в методе гармоник.

\subsection{Решение разностной задачи.}
Используемую разностную схему легко переписать в следующем виде
$$\Lambda\hat{y} = \frac{1}{2}((y_{t})_{j+1} + (y_{t})_{j-1})$$
где $\Lambda$ - оператор Лапласа, а $(y_{t})_{j+1}$ и $(y_{t})_{j-1}$ - это аппроксимации производной по времени соответственно в двух крайних левых точках шаблона и в двух крайних правых.

Рассматриваемая схема аппроксимирует уравнение теплопроводности.
$$u_{t} = u_{xx}$$

Предполагается, что мы решаем задачу в некоторой(прямоугольной) области. Тогда мы должны задать начальные условия (в т. $t = 0$) и граничные условия (в т. $x = 0$ и $x = L$). 

Рассмотрим например холодный стержень, у которого на левом конце поддерживается постоянная температура.
\begin{align*}
&y(x, 0) = 0, \\
&y(0, t) = y_0,\\
&y(L, t) = 0.
\end{align*}
Значения $y$ на 1-м и всех последующих слоях по времени, рассчитываются по формуле $$B\hat{y} = (B - \tau A)y \quad (*)$$ где справа в равенстве стоит известный вектор.
Для этого используется методом прогонки, который в данном случае будет устойчивым т.к. матрица $B$ имеет диагональное преобладание.

Действительно, для неё: $a1 + a2 + a3 = 1$

\section{Приложение}

Рассмотрим вопрос о положительной определённости матрицы вида  $$\ M  = \begin{vmatrix}
a & \pm1 & 0 & 0&.\\
\pm1 & a & \pm1 & .\\
.&.&.&.&.\\
.&.&0&\pm1&a 
\end{vmatrix}$$
где число $a\geq0$,  иначе матрица положительно определённой не является.

Главные миноры такой матрицы находятся из рекуррентного соотношения 
$$\Delta_k = a\Delta_{k-1} - \Delta_{k-2}, \quad \Delta_0 = 1,\quad \Delta_{-1} = 0 $$

Решим эту рекурренту $$\lambda^2= a\lambda-1$$
$$\lambda_1 = \frac{a+\sqrt{a^2-4}}{2},\quad\lambda_2 = \frac{a-\sqrt{a^2-4}}{2}$$
$$\Delta_k = c_1\lambda_1^k + c_2\lambda_2^k$$

Используя $ \Delta_0 = 1$,\quad$ \Delta_{-1} = 0 $ находим $c_1$ и $c_2$
$$c_1+c_2 = 1, \quad \frac{c1}{c2}= -\frac{\lambda_1}{\lambda_2}$$
 $$c_1 =\frac{\lambda_1}{\lambda_1-\lambda_2},  \quad c_2 = \frac{\lambda_2}{\lambda_2-\lambda_1}$$
 
По Критерию Сильвестра $M > 0$ т. и т.т, когда все  $\Delta_k > 0$.
Найдём ограничения на $a$ для выполнения этого равенства при всех k.

\begin{flushleft}
1. a > 2;
\end{flushleft}
$$
\frac{\lambda_1^{k+1}}{\lambda_1-\lambda_2} + \frac{\lambda_2^{k+1}}{\lambda_2-\lambda_1} > 0$$
$$\lambda_1^{k+1} - \lambda_2^{k+1}>0$$
$$\lambda_1>\lambda_2$$
что выполнено при всех рассматриваемых a.

\begin{flushleft}
2. a < 2;
\end{flushleft}
$$\Delta_k = \frac{\lambda_1^{k+1}}{\lambda_1-\bar{\lambda}_1} - \frac{\bar{\lambda}_1^{k+1}}{\lambda_1-\bar{\lambda}_1}, \quad \text{т. к.}\quad\lambda_2 = \bar{\lambda}_1\text{(комплексно сопряжены)}$$
подставляя $\lambda_1 = |r|e^{i\varphi}$, получаем 
$$\Delta_k = \frac{|r|}{Im\lambda_1}\frac{e^{i(k+1)\varphi} - e^{-i(k+1)\varphi}}{2i} = \frac{|r|}{Im\lambda_1}\sin{(k+1)\varphi}$$
видим, что в этом случае $\Delta_k < 0$ при некоторых $k$.

\begin{flushleft}
3. a = 2;
\end{flushleft}
$$\Delta_k = c_1 k + c_2$$
$$c_2 = 1,\quad c_1 = 1 $$
$$\Delta_k = k + 1 > 0, \quad \text{при всех } k > 0$$
Обобщая всё результаты получаем условия на $a$: $a\geq2$

Пользуясь этим легко найти условие положительной определённости для матрицы вида $$\ M^{'}  = \begin{vmatrix}
a &  b & 0 & 0&.\\
b & a & b & .\\
.&.&.&.&.\\
.&.&0& b &a 
\end{vmatrix}$$
где $b$ уже произвольное число.

Если $b = 0$ проверка очевидна, если $b\neq0$ получаем
$$\ M^{'}  = |b|\begin{vmatrix}
\frac{a}{|b|} &  sign(b) & 0 & 0&.\\
sign(b) & \frac{a}{|b|} & sign(b) & .\\
.&.&.&.&.\\
.&.&0& sign(b) &\frac{a}{|b|} 
\end{vmatrix}$$
А из доказанного следует, что $M^{'}$ положительно определена т. и т.т. когда $a\geq2|b|$
Отметим, что это условие означает диагональное преобладание матрицы $M^{'}$ при положительности диагональных элементов.
\end{document}
